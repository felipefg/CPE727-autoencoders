\documentclass{beamer}
\usepackage{amsfonts,amsmath,oldgerm}
\usepackage{tikz}
\usepackage[portuguese]{babel}
\usepackage{booktabs}
\usetikzlibrary{arrows.meta,calc, positioning}
\usetheme{sintef}

\newcommand{\testcolor}[1]{\colorbox{#1}{\textcolor{#1}{test}}~\texttt{#1}}

\usefonttheme[onlymath]{serif}

\titlebackground*{assets/background}

% adicionar o numero na lista final da apresentação
\setbeamertemplate{bibliography item}{\insertbiblabel}

\newcommand{\hrefcol}[2]{\textcolor{cyan}{\href{#1}{#2}}}

\title{Autoencoders - Testes e Aplicações}
\subtitle{}
\course{CPE 727 - Aprendizado de Profundo}
\author{Felipe Fink Grael, Rafael Tadeu Cardoso dos Santos, Thalles Nonato Leal Santos e Jefferson Osowsky}
\begin{document}
\maketitle

% ============================
% Autoencoder: Motivação
% ============================

% \section{Autoencoder Simples}

\section{Denoising Autoencoder}

\begin{frame}{Dataset}
    \begin{figure}
        \centering
        \includegraphics[width=0.6\textwidth]{assets/mnist_dataset_example_grid.png}
        \caption{Imagens do MNIST e Noisy MNIST. Ruído Gaussiano com 0.3 de desvio padrão.}
    \end{figure}
\end{frame}

\begin{frame}{Modelo}
    \begin{columns}
        \begin{column}{0.5\textwidth}
            \begin{itemize}
                \item Input: imagens 28x28x1 (Noisy MNIST)
                \item Encoder:
                    \begin{itemize}
                        \item 3 Camadas Convolucionais (Kernel 3x3 e Stride 1)
                        \item BatchNorm
                        \item ReLU
                        \item MaxPooling(Kernel 2x2, Stride 2)
                    \end{itemize}
                \item Decoder:
                    \begin{itemize}
                        \item 2 Camadas Convolucionais T. (Kernel 3x3 e Stride 2)
                        \item BatchNorm
                        \item ReLU
                        \item Camada  Convolucional T. Final (Kernel 3x3, Stride 1)
                        \item Sigmoid
                    \end{itemize}
            \end{itemize}
        \end{column}
        \begin{column}{0.5\textwidth}
            \begin{figure}
                \centering
                \includegraphics[width=0.75\textwidth]{assets/dae_architecture.png}
            \end{figure}
        \end{column}
    \end{columns}
\end{frame}

\begin{frame}{Metodologia}
    \begin{figure}
        \centering
        \includegraphics[width=0.9\textwidth]{assets/metodologia_dae.png}
    \end{figure}
\end{frame}

\begin{frame}{Treinamento}
    \begin{figure}
        \centering
        \includegraphics[width=0.8\textwidth]{assets/fold_0_training_curve.png}
    \end{figure}
\end{frame}

\begin{frame}{Resultado - Denoising}
    \begin{figure}
        \centering
        \includegraphics[width=0.6\textwidth]{assets/denoising_comparison_all_methods_0.3.png}
    \end{figure}
\end{frame}

% \section{Variational Autoencoder}
%
\section{Reuter Corpus com Autoencoder}
\begin{frame}{Reuter Corpus V1}
    \begin{itemize}
        \item Coleção de 804.414 artigos de notícias da Reuters (1996-1997)
        \item 103 categorias de tópicos organizadas hierarquicamente
        \item Classificação multi-label: cada documento pode pertencer a múltiplas categorias
        \item Categorias abrangem temas como economia, política, esportes, tecnologia, etc.
    \end{itemize}
\end{frame}

\begin{frame}{Método}
    \begin{itemize}
        \item Objetivo: Replicar o plot do Hinton (2006)
        \item Autoencoder com mesma arquitetura (2000, 500, 250, 125, 2)
        \item Ativação linear na camada do espaço latente, e Leaky ReLU nas demais
        \item Otimização com AdamW, sem pré-treino
    \end{itemize}
\end{frame}

\begin{frame}{Evolução do treino}
    \begin{figure}
        \centering
        \includegraphics[width=0.8\textwidth]{assets/rcv1_train.png}
    \end{figure}
\end{frame}

\begin{frame}{Resultado}
    \begin{figure}
        \centering
        \includegraphics[width=0.8\textwidth]{assets/rcv1_scatter.png}
    \end{figure}
\end{frame}

\section{Anexos}

\begin{frame}{Treinamento}
    \begin{figure}
        \centering
        \includegraphics[width=0.8\textwidth]{assets/fold_1_training_curve.png}
    \end{figure}
\end{frame}
\begin{frame}{Treinamento}
    \begin{figure}
        \centering
        \includegraphics[width=0.8\textwidth]{assets/fold_2_training_curve.png}
    \end{figure}
\end{frame}
\begin{frame}{Treinamento}
    \begin{figure}
        \centering
        \includegraphics[width=0.8\textwidth]{assets/fold_3_training_curve.png}
    \end{figure}
\end{frame}
\begin{frame}{Treinamento}
    \begin{figure}
        \centering
        \includegraphics[width=0.8\textwidth]{assets/fold_4_training_curve.png}
    \end{figure}
\end{frame}
\begin{frame}{Resultado - Denoising}
    \begin{figure}
        \centering
        \includegraphics[width=0.6\textwidth]{assets/denoising_comparison_all_methods_0.5.png}
    \end{figure}
\end{frame}
\begin{frame}{Resultado - Denoising}
    \begin{figure}
        \centering
        \includegraphics[width=0.6\textwidth]{assets/denoising_comparison_all_methods_0.8.png}
    \end{figure}
\end{frame}

\backmatter
\end{document}
